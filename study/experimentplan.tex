\documentclass[11pt]{article}
\usepackage{times}
\usepackage{fullpage}
\usepackage{amsmath,amsthm,amssymb}
\usepackage{array}
\usepackage{enumitem}
\usepackage{fancyvrb,moreverb}


\DefineVerbatimEnvironment{code}{Verbatim}
{fontsize=\small}%,frame=single}

\newcommand{\codelist}[1]{{\footnotesize\listinginput[5]{1}{#1}}}
\newcommand{\event}[1]{\textit{#1}}
\newcommand{\sched}[1]{\item[\textit{#1}]}

\newcommand{\choices}[1]{\vspace{-1ex}{\small#1}}%{\centering\fbox{#1}}}
\newcommand{\Likert}{\choices{strongly agree $|$ agree $|$
		do not agree or disagree $|$ disagree $|$ strongly disagree}}
\newcommand{\SemDiff}{\choices{much easier in CPP $|$ easier in CPP $|$
		same difficulty in both $|$ easier in prototype $|$
		much easier in prototype}}



\begin{document}
	
	\title{Understanding Software Variation: Experiment Plan}
	\author{Miles Van de Wetering and Eric Walkingshaw}
	\date{}
	\maketitle
	
	\section{Experiment Goals}
	\label{sec:goals}
	
	Software variation is often difficult to comprehend - even more abstract than conditional execution, variational programming changes the very code generated at compile time. it is also very important to understand - variation is used in widespread production software. We are expanding upon previous efforts to design an editor that supports a visual representation of variation. In this study, we will focus on two different metrics: how easy it is for a user to \emph{understand} variation in existing code, and how easy it is for a user to \emph{correctly modify} variation. We will measure these two concepts through a variety of goals:
	%
	
	\begin{enumerate}[label=Goal \arabic*:,leftmargin=*]%labelindent=\parindent]
		
		\item Determine whether users can \emph{more accurately} deduce the
		\emph{number of variants} represented in code presented in our prototype
		compared to code annotated with CPP.
		
		\item Determine whether users can \emph{more quickly} deduce the \emph{number
			of variants} represented in code presented in our prototype compared to code
		annotated with CPP.
		
		\item Determine whether users can conduct a \emph{feature subtraction} operations more quickly and accurately.
		
		\item Determine whether users can conduct a \emph{feature addition} operations more quickly and accurately.
		
		\item Determine whether users consider the prototype to be \emph{more understandable} than code annotated with CPP.
		
		\item Determine how users' \emph{execution comprehension} (ability to reason about a program) is affected.
		
	\end{enumerate}
	
	
	\section{Participants}
	\label{sec:participants}
	
	For our experiment we plan to recruit at least 35 Computer Science
	undergraduate students (the minimum of 30 required for statistical
	significance, plus 5\ in case of no-shows).  Subjects will be recruited through
	BeaverSource and the EECS mailing list.  Participants will be compensated \$20.
	
	Potential subjects must take a brief screening test before signing up for the
	study.  This will be used to confirm a basic understanding of C and CPP, and
	only students that pass the screening test will be asked to take part in the
	study.
	%
	We anticipate recruiting problems with such strict prerequisites.  If the
	screening procedure eliminates too many students or too few attempt to register
	in the first place, we will expand our population to include non-Computer
	Science students with programming experience (such students could be found, for
	example, through the Linux Users mailing list).  If we still do not have enough
	participants, we will attempt to simplify the screening test and extend the
	tutorial accordingly.
	
	Our experiment will be performed within-subjects, so all participants will
	undergo all treatments.  The ordering of tasks will be randomized, however,
	along with the tasks themselves (see Section~\ref{sec:design}).  All
	participants in the same experiment session will perform the tasks in the same
	order, and the tasks and ordering of tasks will be randomized per experiment session.
	
	
	\section{Experiment Materials}
	\label{sec:materials}
	
	Prior to participating in the experiment, potential subjects will submit a
	registration form containing a screening test.  The registration form will
	collect the student's name, email address, phone number, major, year of study,
	and two questions confirming that the student has a basic understanding of C
	and CPP.  The accompanying screening test is designed to confirm the student's
	responses to these questions and is shown in Figure~\ref{fig:screen}.
	
	\begin{figure}
		\centering
		\begin{tabular}{|m{10cm}|m{3cm}|}
			\hline
			The return value of the C program at right depends on whether or not the C
			Preprocessor macros \texttt{A} and \texttt{B} are defined when the code is
			compiled.  For each question below, write the return value if the code is
			compiled with the given macro settings.
			%
			\begin{itemize}[itemsep=1em]
				\item A = undefined, B = undefined: 
				\item A = undefined, B = defined: 
				\item A = defined, B = undefined: 
				\item A = defined, B = defined: 
			\end{itemize}
			&
			\begin{code}
				int main() {
					#ifdef A
					int x = 0;
					#else
					int x = 1;
					#endif 
					if (x) {
						#ifdef B
						return 3;
						#else
						return 4;
						#endif
					}
					return 5;
				}
			\end{code}
			\\ \hline
		\end{tabular}
		\caption{C and CPP knowledge screening test.}
		\label{fig:screen}
	\end{figure}
	
	The prototype tool will be implemented as an Atom plugin using Javascript.
	The experiments will be administered in a lab setting, on provided computers
	(20\% more computers than the expected number of participants in each session,
	in case of technical difficulties), via the Atom editor plugin.  The CPP annotated code
	will be presented in it's own file, and will include syntax
	highlighting.  The prototype tool will be implemented using HTML and
	Javascript.  Questions will be presented to the user in a separate online survey which they will progress through after completing each task..
	
	The code that will be used in the tasks themselves is included in the Appendix.
	There are three examples: one implementing playing cards, one implementing an
	alarm clock, and one implementing the Fibonacci sequence.  One of these will be
	used in the tutorial (currently, the alarm clock), and the other two will be
	used in the tasks.  Because one example may be inherently more confusing than
	another, we will make prototype and CPP versions of both remaining examples and
	randomly determine which combination is used for each session.
	
	At the beginning of the experiment session, a tutorial
	will be verbally administered.  Users will have minor tasks to perform during
	this tutorial.  These will be done on the same computers and in the same way
	as the subsequent tasks.
	
	
	\subsection{Logging}
	\label{sec:logging}
	
	The local web servers will produce log files which contain the data necessary
	for our analysis.  Table~\ref{tbl:cpplog} describes the structure of the log
	file that will be generated during CPP related tasks.  Because our interface
	for these tasks is so simple, there are only two possible events.  A
	\event{QuestionStarted} event occurs when a subject loads a page containing a
	new question, and an \event{AnswerSubmitted} event occurs when the subject
	submits the form containing the answer.  The \event{AnswerSubmitted} event
	contains the answer the subject entered, which we will use to determine the
	accuracy of the subject's understanding.  Both events contain timestamps, which
	we can take the difference of to determine the amount of time it took to answer
	the question.  In order for this strategy to work effectively, we will need to
	precede each question page with a ``Start Question'' page that both reminds the
	user that the questions are timed, and gives them an un-timed place to rest
	briefly.
	
	\begin{table}
		\centering
		\begin{tabular}{|l | p{.4\textwidth}|}
			\hline
			\textbf{Field} & \textbf{Description} \\
			\hline
			TimeStamp  & Time that the event occurred. \\
			SubjectID  & Unique subject identifier. \\
			TaskID     & Task number. \\
			QuestionID & Question number within a task. \\
			EventType  & \event{QuestionStarted} or \event{AnswerSubmitted}. \\
			Answer     & The submitted answer, if applicable. \\
			\hline
		\end{tabular}
		\caption{Structure of log file for CPP-related tasks.}
		\label{tbl:cpplog}
	\end{table}
	
	Table~\ref{tbl:protolog} describes the structure of the log file that will be
	generated during the prototype related tasks.  The structure is almost
	identical to the log file for CPP tasks, except that we also add a
	\event{TagSelected} task that is logged whenever a user selects a new tag in
	the prototype interface, changing the currently visible variant.  When this
	event occurs, we also record the new set of currently visible tags.  Strictly
	speaking, only the just-selected tag is needed to replay the subject's actions,
	but we record all currently visible tags just to provide a bit of redundancy in
	case the user manages to get into an unexpected state (for example, by using
	the browser's ``Back'' button, although we will probably try to disable this
	particular feature).
	
	\begin{table}
		\centering
		\begin{tabular}{|l | p{.5\textwidth}|}
			\hline
			\textbf{Field} & \textbf{Description} \\
			\hline
			TimeStamp  & Time that the event occurred. \\
			SubjectID  & Unique subject identifier. \\
			TaskID     & Task number. \\
			QuestionID & Question number within a task. \\
			EventType  & \event{QuestionStarted}, \event{AnswerSubmitted},
			or \event{TagSelected}. \\
			Parameter  & If \event{AnswerSubmitted}, the submitted answer. \\
			& If \event{TagSelected}, the new set of selected tags. \\
			\hline
		\end{tabular}
		\caption{Structure of log file for prototype-related tasks.}
		\label{tbl:protolog}
	\end{table}
	
	\subsection{Post-Task Questionnaire}
	\label{sec:post}
	
	Finally, a post-task questionnaire will be given so that we can assess the
	perceived usefulness of the prototype compared to CPP annotations.  This
	questionnaire is provided in Figure~\ref{sec:post}.
	%
	We begin by asking a few questions about the tasks themselves.  Knowing how
	difficult the users found the tasks will help us to interpret our other
	results.  We then ask the subjects to directly compare the two systems for a
	few specific understanding tasks. 
	%
	While we use a Likert scale for several questions in the questionnaire, we use
	a semantic differential scale for the direct comparison questions, in order to
	avoid as much bias as possible.  We then ask some questions specific to one
	system or the other, and some questions which try to determine whether or not
	the subject believes they have a grasp on the concept of dimensions.  Finally,
	we end with gusto, with a basic ``which is better'' question, in which we force
	the user to commit to one of the systems. 
	
	\begin{figure}
		
		\textbf{Post-Task Questionnaire}
		
		Subject ID:
		
		$ $
		
		Please circle your response to each question.
		
		\begin{enumerate}%[itemsep=0.5ex]
			
			\item In general, the tasks I completed in this experiment were difficult.
			
			\Likert
			
			\item In which system was it easier to determine \emph{how many} different
			programs were represented by the code?
			
			\SemDiff
			
			\item In which system was it easier to understand \emph{a particular} program
			generated from the code?
			
			\SemDiff
			
			\item In which system was it easier to see how the different programs were
			\emph{related} to each other?
			
			\SemDiff
			
			\item When looking at the CPP annotated code, I could tell which macros were
			meant to be mutually exclusive (only one can be selected at a time).
			
			\Likert
			
			\item I think that I could translate the CPP annotated code into an equivalent
			representation in the prototype.
			
			\Likert
			
			\item In general, I found the CPP annotated code easy to understand.
			
			\Likert
			
			\item I think that I could translate the code in the prototype into an
			equivalent CPP annotated program.
			
			\Likert
			
			\item In general, I found the prototype easy to understand.
			
			\Likert
			
			\item Overall, which tool do you think makes it easier to understand software
			variation (code that represents many different programs)?
			
			\choices{CPP $|$ the prototype}
			
		\end{enumerate}
		
		\caption{Post-Task Questionnaire}
		\label{fig:post}
	\end{figure}
	
	%\noindent
	%For the previous questions there was very little at stake, so we were not too
	%concerned about influencing our results with the wording of our questions.
	%Next we ask the subjects to directly compare the two systems for a few specific
	%tasks, however.  It seems that a Likert scale is not appropriate here since it
	%would force us to commit to one system as the positive/agree system.
	%%
	%Instead, subjects will answer each question according to the following
	%five-point, semantic differential scale: ``[Much] Easier in CPP'', ``Same
	%Difficulty in Both'', ``[Much] Easier in Prototype''.
	
	%\noindent
	%Next we ask some questions specific to the CPP annotated code, and in
	%particular about whether or not the notion of dimensions made sense.  Then some
	%mostly symmetric questions about the prototype.  We switch back to the Likert
	%scale.
	
	%\noindent
	%And finally, we ask the big question, forcing users to pick from only two
	%responses: CPP or the prototype.
	
	
	\section{Tasks}
	\label{sec:tasks}
	
	Each session will consist of two tasks: one CPP task, and one prototype task.
	Throughout the CPP task, subjects will be shown one of the pieces of code from
	the Appendix in a web browser window, with syntax highlighting.  The task
	consists of answering several questions as quickly and accurately as possible.
	The code is displayed only while answering a question.  In between questions,
	subjects will be shown a wait screen that reminds them (subtly) that responses
	are timed, and prompts them to click a link to start the next question when
	they are ready.
	
	The prototype task proceeds similarly, except that the HTML/Javascript
	prototype replaces the static code in the question-answering window.  Since we
	will be comparing the accuracy and speed of the subjects' answers in both
	tasks, the questions will of course be mostly the same.  However, there are
	some questions which are specific to the CPP task.  These do not relate
	directly to the primary research questions presented here, but will be helpful
	for our secondary research questions described elsewhere.  Questions will also
	sometimes require minor rewording between tasks (for example, changing ``CPP''
	to ``the prototype'').
	%
	Finally, questions will sometimes be phrased such that they answer a previous
	question in the sequence.  This will hopefully prevent one incorrect answer
	from snowballing into several.
	
	Below is a sequence of questions that will be provided for the playing cards
	example used for the CPP task.  These questions can be easily adapted to the
	other example and task.  Questions that are CPP-specific will be marked with an
	asterisk (*).
	
	\begin{enumerate}[label=Q\arabic*:,leftmargin=*]%labelindent=\parindent]
		
		\item How many variants can be generated from this code?
		
		\item Suppose that we fix the \texttt{AcesLow} macro to be defined, how many
		variants can be generated given this constraint?
		
		\item How many different CPP macros are present within this code?*
		
		\item Given that there are three different macros, and that macros can be
		either defined or undefined, how many different ways can these macros be set at
		compile time?*
		
		\item Because there are six different ways to assign these macros, there are
		a total of six different variants that can be generated from this code.  How
		many do you think the programmer \emph{intended} to define?  Why?  (Open
		answer, untimed).*
		
		\item What is the output of this program if the \texttt{AcesLow} and
		\texttt{Verbose} macros are defined, but the \texttt{AcesHigh} macro is
		undefined?
		
		\item What is the output of this program if the \texttt{AcesLow} macro is
		defined, but the \texttt{Verbose} and \texttt{AcesHigh} macros are undefined?
		
		\item What is the output of this program if the \texttt{AcesHigh} and
		\texttt{Verbose} macros are defined, but the \texttt{AcesLow} macro is
		undefined?
		
	\end{enumerate}
	
	
	\section{Hypotheses and Variables}
	\label{sec:hypotheses}
	
	Following are the semi-formalized and English versions of the null and
	alternative hypotheses that follow from our listed goals in
	Section~\ref{sec:goals}.  For each Goal $i$, the corresponding null hypothesis
	is labeled $H_{i0}$, and the corresponding alternative hypothesis is labeled
	$H_{i1}$.  We use $a_p(\cdot)$ to represent the average accuracy of a
	prediction $p$ and $t_p(\cdot)$ to represent the average question response time
	for a prediction $p$.  Predictions can be either $n$, for the number of
	variants represented in some code, or $b$, for the predicted behavior of some
	particular variant.  As arguments to $a$ and $t$, we use $C$ to represent
	questions about code represented in CPP, and $P$ to represent questions about
	code represented in the prototype.  Finally, we use comparison operators like
	$<$ as shorthand for phrases like, ``is statistically significantly less
	than'', and $u(\cdot)$ to represent the subjects post-experiment rating of the
	understandability of each tool.
	
	\begin{enumerate}[leftmargin=*,labelindent=\parindent]
		
		\item[$H_{10}$:]
		$a_n(P) \not> a_n(C)$.
		Users \emph{do not} predict the number of variants more
		accurately when using the prototype than when looking at CPP-annotated code.
		
		\item[$H_{11}$:]
		$a_n(P)     > a_n(C)$.
		Users predict the number of variants more
		accurately when using the prototype than when looking at CPP-annotated code.
		
		\item[$H_{20}$:]
		$t_n(P) \not< t_n(C)$.
		Users \emph{do not} predict the number of variants in less
		time using the prototype than when looking at CPP-annotated code.
		
		\item[$H_{21}$:]
		$t_n(P)     < t_n(C)$.
		Users predict the number of variants in less
		time using the prototype than when looking at CPP-annotated code.
		
		\item[$H_{30}$:]
		$a_b(P) \not> a_b(C)$.
		Users \emph{do not} describe the behavior of a particular
		variant more accurately when using the prototype than when looking at
		CPP-annotated code.
		
		\item[$H_{31}$:]
		$a_b(P)     > a_b(C)$.
		Users describe the behavior of a particular
		variant more accurately when using the prototype than when looking at
		CPP-annotated code.
		
		\item[$H_{40}$:]
		$t_b(P) \not< t_b(C)$.
		Users \emph{do not} describe the behavior of a particular
		variant in less time when using the prototype than when looking at
		CPP-annotated code.
		
		\item[$H_{41}$:]
		$t_b(P)     < t_b(C)$.
		Users describe the behavior of a particular
		variant in less time when using the prototype than when looking at
		CPP-annotated code.
		
		\item[$H_{50}$:]
		$u(P)   \not> u(C)  $.
		Users \emph{do not} rate the prototype as more understandable
		than code annotated with CPP.
		
		\item[$H_{51}$:]
		$u(P)       > u(C)  $.
		Users rate the prototype as more understandable
		than code annotated with CPP.
		
	\end{enumerate}
	
	The notation described and used above reveals the structure of the major
	variables in our experiment.  The dependent variables---accuracy, response
	time, understandability rating---are represented as functions of the
	independent variables of the treatment group (CPP or prototype) and task type
	(variant counting or understanding).  Other major independent variables not
	reflected in the above equations are the example used for each treatment group
	(Fibonacci or playing cards) and the order in which the treatment groups are
	presented (CPP first or prototype first).  In Section~\ref{sec:design} we will
	show how we use randomization to hopefully mitigate the effects of these
	``uninteresting'' independent variables.
	
	% \begin{align*}
	% H_{01} &= a_n(P) \not> a_n(C) \\
	% H_{11} &= a_n(P)     > a_n(C) \\
	% H_{02} &= t_n(P) \not< t_n(C) \\
	% H_{12} &= t_n(P)     < t_n(C) \\
	% H_{03} &= a_b(P) \not> a_b(C) \\
	% H_{13} &= a_b(P)     > a_b(C) \\
	% H_{04} &= t_b(P) \not< t_b(C) \\
	% H_{15} &= t_b(P)     < t_b(C) \\
	% H_{05} &= u(P)   \not> u(C)   \\
	% H_{15} &= u(P)       > u(C)   \\
	% \end{align*}
	
	% \begin{enumerate}[leftmargin=*,labelindent=\parindent]
	% 
	% \item[$H_{01}$:] Users \emph{do not} predict the number of variants more
	% accurately when using the prototype than when looking at CPP-annotated code.
	% 
	% \item[$H_{11}$:] Users predict the number of variants more
	% accurately when using the prototype than when looking at CPP-annotated code.
	% 
	% \item[$H_{02}$:] Users \emph{do not} predict the number of variants in less
	% time using the prototype than when looking at CPP-annotated code.
	% 
	% \item[$H_{12}$:] Users predict the number of variants in less
	% time using the prototype than when looking at CPP-annotated code.
	% 
	% \item[$H_{03}$:] Users \emph{do not} describe the behavior of a particular
	% variant more accurately when using the prototype than when looking at
	% CPP-annotated code.
	% 
	% \item[$H_{13}$:] Users describe the behavior of a particular
	% variant more accurately when using the prototype than when looking at
	% CPP-annotated code.
	% 
	% \item[$H_{04}$:] Users \emph{do not} describe the behavior of a particular
	% variant in less time when using the prototype than when looking at
	% CPP-annotated code.
	% 
	% \item[$H_{15}$:] Users describe the behavior of a particular
	% variant in less time when using the prototype than when looking at
	% CPP-annotated code.
	% 
	% \item[$H_{05}$:] Users \emph{do not} rate the prototype as more understandable
	% than code annotated with CPP.
	% 
	% \item[$H_{15}$:] Users rate the prototype as more understandable
	% than code annotated with CPP.
	% 
	% \end{enumerate}
	
	\section{Experiment Design}
	\label{sec:design}
	
	Our experiment will be conducted \emph{within subjects} to maximize statistical
	power with an expected small number of subjects.
	
	All uninteresting/potentially confounding independent variables will be
	distributed and randomized as much as possible.  Specifically, we will have
	four different groups representing the four possible ways of instantiating our
	two most significant uninteresting independent variables: the example used in each
	treatment group (Fibonacci or playing cards), and the order that the treatment
	groups are presented (CPP first or prototype first).  The four possibilities
	are represented in Table~\ref{tbl:groups}.
	
	\begin{table}[h]
		\centering
		\begin{tabular}{|c|c c|}
			\hline
			& Fibonacci & Cards     \\
			\hline
			First Task  & CPP       & CPP       \\
			Second Task & Prototype & Prototype \\
			\hline
		\end{tabular}
		\caption{Four experiment groups, to distribute potentially confounding variables.}
		\label{tbl:groups}
	\end{table}
	
	For pragmatic reasons, we will assign all subjects in a particular session to
	the same experiment group.  Therefore, we will aim to have either four or eight
	sessions in order to distribute subjects across the four groups evenly.  We
	will randomly assign each groups to sessions within these constraints.
	
	All other potentially confounding variables, such as differing experience
	levels and intelligence, should be mitigated by randomly assigning subjects to
	sessions, by the screening test, and by treating all subjects with the same
	introductory tutorial.
	
	
	\section{Experiment Procedure}
	\label{sec:procedure}
	
	Subjects will be sent an email reminder one day before their scheduled session.
	Subjects will also be called the evening before their scheduled session to
	confirm their participation.
	%
	Two hours are allotted for each experiment session, scheduled as follows.
	
	\begin{itemize}[leftmargin=3\parindent]
		
		\sched{0:00} Participants arrive at the testing room and are randomly assigned
		to a computer.  Driver/helper enters subjects' IDs into computer as they are
		seated.  Tutorializer calls participants that have not arrived by the scheduled
		start time to confirm that they are still coming.
		
		\sched{0:05} Door closed.
		
		\sched{0:05--0:20} Informed consent form read and signed.  Confirm that
		subjects' IDs are on consent forms.
		
		\sched{0:20--0:40} Give tutorial (tutorializer: Eric, driver: Duc).
		
		\sched{0:40--0:50} Questions and extra time in case of technical problems.
		
		\sched{0:50--0:55} Introduction to Task 1.
		
		\sched{0:55--1:10} Task 1.
		
		\sched{1:10--1:15} Introduction to Task 2.
		
		\sched{1:15--1:30} Task 2.
		
		\sched{1:30--1:40} Post-task questionaire.
		
		\sched{1:40--2:00} Pay subjects and get receipts.
		
		\sched{2:00} Retrieve log files and get final screenshots.
		
	\end{itemize}
	
	More specific details of the procedure are described throughout this plan, but
	particularly in Section~\ref{sec:materials}, which describes how and what data
	is captured in log files and the post-task questionnaire.
	
	
	\section{Analysis Procedure}
	\label{sec:analysis}
	
	The data that is required to evaluate each of our hypotheses follows from the
	formalized hypotheses provided in Section~\ref{sec:hypotheses} and the
	discussion of experiment materials in Section~\ref{sec:materials}.  For each
	dependent variable in the hypotheses, we can trace a path back to the data that
	allows us to measure that variable.  The results are summarized in
	Table~\ref{tbl:data}.
	
	\begin{table}[h]
		\centering
		\begin{tabular}{|l | l | m{8.5cm} |}
			\hline
			\textbf{Hypotheses} & \textbf{Dependent variable} & \textbf{Data source}     \\
			\hline
			$H_{10},H_{11},H_{30},H_{31}$ & Answer accuracy   & Log file: answers in \event{AnswerSubmitted} events \\
			$H_{20},H_{21},H_{40},H_{41}$ & Response time     & Log file: difference in TimeStamp of corresponding \event{QuestionStarted} and \event{AnswerSubmitted} events \\
			$H_{50},H_{51}$               & Understandability & Post-experiment questionnaire: answers to question 11 \\
			\hline
		\end{tabular}
		\caption{Data sources to evaluate each hypothesis.}
		\label{tbl:data}
	\end{table}
	
	\noindent
	We have not yet considered (nor discussed in class!) how to evaluate these
	hypotheses statistically.
	
	
	\section*{Appendix: Code Listings}
	
	\subsection{cards.c}
	\label{sec:cards}
	\codelist{./comprehension.c}
	
	\subsection{clock.c}
	\label{sec:clock}
	\codelist{../Examples/clock/tutorial.c}
	
	\subsection{fib.c}
	\label{sec:fib}
	\codelist{../Examples/fibonacci/fib.c}
	
\end{document}
